Syllabus:
\begin{itemize}
\item Descriptive versus Inferential Statistics, Data Collection and Sampling, Probability distributions, Model building and experimental design, Hypothesis testing, Analysis of Variance, Regression and Time series analysis, Factor analysis.
\item The power, elegance and limitations of statistical/mathematical modelling (AckoffÆs Systems Thinking, G÷delÆs Incompleteness Theorem, TuringÆs Computability Concepts, NashÆs Equilibrium Theory).
\item How to construct a sample.  The use of sampling theory and methods to identify appropriate sampling frames and the need to calculated the power of a test .
\item The use of Distribution Theory for the analysis and interpretation of large volumes of experimental data.  Including the concept of a parameter estimation, maximum likelihood , the method of moments and Cramer-Rao Uniqueness Theorem.     
\item The uses of multiple regression analysis: logit, probit, poisson etc. discriminant analysis, and linear structural equations in research.  

\item To highlight the emphasis on the mean within classical statistics and to introduce students to robust methodologies e.g. median polish and M - estimators etc. outliers, influence points and Robust Bisquare Regression
\item Experimental Designs; Response Surface Methods: Method of Steepest Ascent, verifying adequacy of first order model. Analysis of second order response surface.
\end{itemize}
