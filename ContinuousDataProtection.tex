
\documentclass[12pt]{article}

%opening
\title{Data Science}
\author{Kevin O'Brien}

\begin{document}
	\section*{Continuous Data Protection}
	
\begin{itemize}
\item Continuous data protection is a type of backup system that allows for information on a computer system to be constantly backed up on a different server, possibly even at a different physical location. Many consider this to be one of the safest forms of backup protection. While some systems back up the information stored on the main computer every day or even every few days, continuous data protection ensures that not even a day’s work is lost in the event of a catastrophic computer failure.

\item In most cases, continuous data protection works by saving an exact copy of a file a user is working with to a different location. Each time the user changes the file, a new file is saved remotely, usually overwriting the previous file saved under that same name. Thus, it is not real-time backup in that each keystroke or move of the mouse is recorded and saved. That may be a very important distinction for those who are expecting to recover lost work because of things like power failures, where a computer will shut down unexpectedly. Real-time backup products are also available, for those who want them.

\item Computer users can utilize continuous data backup in a number of different ways. Software products currently on the market allow users to save backup settings so that each file saved will go automatically to a backup system. Those who do not want the extra expense of more servers, can utilize an Internet service that offers continuous data protection. In such cases, each file saved is sent through an Internet connection to the service’s servers.

\item If using an Internet service that provides continuous data protection, there is usually a monthly or yearly charge for the service. Also, some companies may only allow a certain amount of space, or at least have tiered pricing levels, so that customers can choose the amount of space they will need. That allows users to have a certain amount of control over their pricing, but that could also lead to higher charges or loss of data if users go beyond their maximum storage level allowed.

\item For those who want the ultimate in protection using a real-time process, it may be best to consider a continuous data protection system with a uninterruptible power supply. This allows the computer to maintain power in the event the regular power has gone off. While these power supplies generally only have a limit of a few minutes or hours, it should be enough time to save all critical files to the data protection system, and shut down the machines safely.
\end{itemize}
\end{document}
